%        File: progress_report_2015-Dec.tex
%     Created: Di Dez 08 09:00  2015 C
% Last Change: Di Dez 08 09:00  2015 C
%
\documentclass[a4paper]{article}
\begin{document}

\title{SIMEX Progress report}
\author{C. Fortmann-Grote}
\date{\today}

\maketitle
\section{Summary of work completed}

\begin{itemize}
  \item Initialized central software repository at github
    (www.github.com/eucall-software)
  \item First version of simulation platform is ready, supports single
    particle coherent diffractive imaging simulations as in simS2E.
  \item Started work on defining interfaces between simulation
    modules to allow coupling of codes from various partner
    institutes.
  \item HZDR have published 'openPMD', an open standard for particle
    and fields data based on hdf5 and adios. Might be used in SIMEX
    to define a common data format if needed.
  \item First VTC took place on Nov. 27, minutes attached.
\end{itemize}

\section{Next steps and TODOs}
\begin{itemize}
  \item Identify required and sensible interfaces (not all modules
    need to be be coupled). Due 12/2015.
  \item Implement in-memory data passing as a requirement for
    handling large PIC simulation data. Due 2/2016.
  \item Implement interface for XFEL source/propagation to Particle-In-Cell
    Photon-Matter-Interaction code (HZDR) as a
    first cross institutional interface. Due 3/2016.
  \item Prepare SIMEX workshop meeting in 3/2016.

\end{itemize}
    \newpage
    \section{Attachments}

\begin{verbatim}
Minutes of VTC EUCALL-SIMEX
Time: Nov. 27th, 12:30 - 13:35 CET
Attendees:
 Carsten Fortmann-Grote (XFEL, video)
 Adrian Mancuso (XFEL, on phone)
 Michael Bussmann (HZDR, on phone)
 Frank Schluenzen (DESY, video)
 Sohichiro Aogaki (ELI-NP, video)
 Marc Glass (ESRF, skype, connected later.)

1.) Presentation "simex_platform, a framework for simulation of photon
experiments" (Carsten)
Comments on the presentation:
*Michael raises the issue of how to license the code, Carsten will look
into lgpl as an option. Refer to HZDR codes for acknowledgements of
contributors.
* Michael: If using python, keep in mind scalability, e.g. for large
data handling (of order 100 TB). Solutions exist.
* Michael: How is the xfel pulse represented? Answer: Wavefront
(intensity+phase) propagation (WPG/SRW) as function or time and place.
Michael: This is ok for converting to representation needed by their PMI.

2.) Overview over relevant software from HZDR:

    * pyDive: Large file data analysis (100 TB)
    * openPMD: open standard for particle and mesh data not a format but
meta file description
    * libSplash : Library for handling hdf/adios files, "speaks" openPMD

3.) Interfaces:
Carsten will prepare documents to collect what physics modules/codes are
available at the various partner institutes. Based on this document, we will
identify the software pieces that need to be coupled in simex_platform.
Not all combinations make sense, Marc questioned if coupling XFEL-Source
to ESRF propagation (ray-tracing)
makes sense. As a guideline we should get a clear view on the kind of
experiments that will be performed at our facilities.


4. ) General discussion:
*Michael suggests looking into existing solutions for coupling software
to each other (workflows).
*Carsten clarifies that the aim is to provide both the framework and the
implementations of physics codes such that users can do simulations with
the existing/provided codes and has the option to hook up their own code.
*Frank comments on build process in simS2E (and partly adopted in
simex_platform), leaves no control to user about what is installed and
how it is configured, might result in conflicting
version requirements.
\end{verbatim}


\end{document}



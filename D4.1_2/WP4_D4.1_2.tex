\documentclass[a4paper]{article}
\usepackage[]{graphicx}
\usepackage{latexsym}
\usepackage[centertags]{amsmath}
\usepackage{amssymb}

%%%%%%%%%%%%%%%%%%%%%%%%%%%%%%%%%%%%%%%%%%%%%%%
%   BIBLIOGRAPHY SETTINGS                     %
%%%%%%%%%%%%%%%%%%%%%%%%%%%%%%%%%%%%%%%%%%%%%%%
\usepackage[bibstyle=nature,sorting=none,=maxnames=1000,eprint=false, defernumbers=true, backend=biber]{biblatex}
\usepackage{hyperref}

\renewcommand*\finalnamedelim{, and\addspace}
\DeclareNameAlias{sortname}{last-first}
\renewcommand{\newunitpunct}{, }

\AtEveryBibitem{%
  \clearfield{day}%
  \clearfield{month}%
  \clearfield{endday}%
  \clearfield{endmonth}%
  \clearfield{issn}%
  \clearfield{issue}%
}
%convert titles to hyperlinks using doi
\ExecuteBibliographyOptions{doi=false}
\newbibmacro{string+doi}[1]{%
  \iffieldundef{doi}{#1}{\href{http://dx.doi.org/\thefield{doi}}{#1}}}
\DeclareFieldFormat*{title}{\usebibmacro{string+doi}{\mkbibemph{#1}}}

%\addbibresource{/home/grotec/Documents/LiteratureDB/bibtex/jabref.bib}
\addbibresource{references.bib}
%%%%%%%%%%%%%%%%%%%%%%%%%%%%%%%%%%%%%%%%%%%%%%%
% END BIBLIOGRAPHY SETTINGS                   %
%%%%%%%%%%%%%%%%%%%%%%%%%%%%%%%%%%%%%%%%%%%%%%%


\title{Design Report and Advanced Simulation Software for Laser -- Matter interaction}
%In fullfilment of deliverables D4.1 and D4.2 in EUCALL SIMEX
\author{Carsten Fortmann-Grote, Axel Huebl, Michael Bussmann } % Add yourself.
\date{\today}

\begin{document}
\maketitle
\section{Introduction}
This document describes the design and implementation of simulation software for interaction of intense, coherent radiation from x-ray free
electron lasers, synchrotrons, and optical lasers with solid, liquid, or gasous samples and targets.
It discusses the generic simulation platform within which the simulations will be performed and the
concrete simulation tools for two virtual experiments:
\begin{enumerate}
  \item Interaction of high power, \textbf{short} pulse optical laser radiation (duration $\approx 10\,\text{fs}$)
    with matter and subsequent probing of the resulting state by intense x-ray pulses,
  \item Interaction of high energy, \textbf{long} pulses of optical laser light (duration $\approx 10\,\text{ps}$
    with a solid target and probing with x-rays.
\end{enumerate}%

\section{The SIMEX simulation platform \label{sec:simex_platform}}
%
\subsection{Introduction}
The computer program \texttt{simex\_platform}
\cite{simex_github}
is a platform for the simulation of experiments at advanced laser light sources. It allows the
user to assemble a virtual experiment through combination of suitable programs for the light source (e.g. a synchrotron, an x-ray free electron
laser or an optical laser), beam transport from the source to the sample or target, interaction of the light with the sample or target,
propagation of the scattered light behind the sample or target, and detection in a light detector. \texttt{simex\_platform} comes preloaded with
a number of such \textit{Calculators}, aimed at the simulation of various typical laser light experiments. Furthermore, researchers
can replace individual built-in \textit{Calculators} by their own codes. In this way, they can embed their codes
into a more realistic simulation environment compared to running the code in stand-alone fashion with more or less idealized parameters.
%
\subsection{Building blocks of a virtual photon experiment}
With slight variations, photon based experiments can be broken down into six individual blocks: The photon source, the photon transport
from the source to the experiment, the interaction of photons and matter, propagation of scattered and transmitted radiation from the
target/sample to the detector and photon detection in a detector, followed by data analysis.
In \texttt{simex\_platform} each of these blocks is
represented by an \textit{Abstract Calculator}, which defines virtual methods for communication between subsequent calculators and execution of
the underlying implementation of a particular algorithm.
These virtual methods can be used e.g. in a workflow manager to execute the virtual experiment.
Specialized \textit{Calculators}, derived from the\textit{Abstract Calculator}, implement one particular stage of the simulation, using a
specific method or algorithm. Data
format conversion, execution statements, and submission of calculation tasks to the operating system, are handled on this
level. The
\textit{Calculators} are the algorithmic building blocks describing subsequent stages in the beamline.
%
\subsubsection{Photon source}
The mechanism to generate the radiation before any optical elements and before any photon-matter interaction has happened.
\begin{description}
  \item[Abstract calculator:] AbstractPhotonSource
  \item[Physics:] Depending on nature of the source: radiation from charged particle acceleration, spontanuous emission, laser medium.
  \item[Input data:] Parameters of the photon source, e.g. for an undulator: electron bunch charge, electron energy, undulator period and length.
  \item[Output data:] Representation of the light source (e.g. as a wavefront, rays, photon distribution)
  \item[Example method:] The code FAST \cite{Saldin1999} generates 3D (x-y-t) wavefronts at the exit of the undulator in an x-ray free electron laser.
\end{description}
%
\subsubsection{Photon Propagation}
Propagates the radiation as described by the photon source calculator from the source to the point of interaction with the sample or target under
    investigation. Describes focussing, filtering, pulse shaping,
    and other optical effects realized through lenses, mirrors, apertures, grids etc.
\begin{description}
  \item[Abstract calculator:] AbstractPhotonPropagator
  \item[Physics:] Propagation of light through optical elements.
  \item[Input data:] Wavefront or rays at beginning of beamline.
  \item[Output data:] Wavefront or rays at target/sample interaction point.
  \item[Example method:] Fourier optics wavefront propagation, ray tracing.
\end{description}
%
\subsubsection{Photon Interactor}
Interaction of the photons with the target or sample. Takes into account elementary processes like absorption, emission, scattering of radiation and secondary processes like collisional ionization and recombination. The end product is the electronic state of the sample/target as a function of time during the interaction with the external light source.
\begin{description}
  \item[Abstract calculator:] AbstractPhotonInteractor
  \item[Physics:] Absorption, emission, and scattering of radiation by charged particles. Secondary processes like electron impact ionization, three
    body recombination. Acceleration of particles in external fields and backreaction of excited matter on the radiation.
  \item[Input data:] Wavefront or rays at sample/target interaction point.
  \item[Output data:] Snapshots of electronic state trajectory during the radiation exposure. E.g. electron density, form factors, electronic
    wavefunction or density matrix.
  \item[Example method:] Molecular Dynamics, Particle-in-cell, Ab-initio.
\end{description}
%
\subsubsection{Photon Diffractor}
Typically in a photon experiment, the scattered or transmitted radiation, or reaction products from the photon-matter
interaction, such as photo-electrons, are detected and used to infer information about the sample/target.
This module is named \textit{Diffractor} although diffraction is by far not the only anticipated diagnostic method to be simulated.
\begin{description}
  \item[Abstract calculator:] AbstractPhotonDiffractor
  \item[Physics:] Absorption, emission, and scattering of radiation by the target/sample.
  \item[Input data:] Wavefront or rays at sample/target interaction point and electronic state trajectory.
  \item[Output data:] Radiation signal at a given position of a radiation detector.
  \item[Example methods:] Born approximation scattering from single molecules, small-angle x-ray scattering (SAXS), x-ray Thomson
    scattering (XRTS).
\end{description}
%
\subsubsection{Photon Detector}
The scattered radiation will ultimately  be collected in a detection device. The Photon Detector models the response of the detector to the incoming
radiation.
\begin{description}
  \item[Abstract calculator:] AbstractPhotonDetector
  \item[Physics:] Response of the detector to incoming radiation, e.g. photoelectron conversion, charge transport, and counting in a photon detector.
  \item[Input data:] Intensity or photon distribution at the detector.
  \item[Output data:] Detector response to the signal, e.g. photons per pixel in an area detector.
  \item[Example method:] X-ray camera simulation toolkit X-CSIT \cite{Joy2015}.
\end{description}
%
\subsubsection{Photon Analysis}
Following the detector readout, the recorded raw data has to be analyzed in order to extract the desired information about the target/sample. The
specific algorithms depend largely on the details of the experiment and the underlying questions. Here, we discuss the analysis steps to be taken
in a single particle diffractive imaging experiment.
\begin{description}
  \item[Abstract calculator:] AbstractPhotonAnalyzer
  \item[Physics:]  Analysis of raw detector data. In SPI, the electron density of the sample is reconstructed by assembling a 3D diffraction volume
    from measured 2D diffraction patterns and subsequent phasing of the data to solve the inverse scattering problem.
  \item[Input data:] Detector response (raw data)
  \item[Output data:] Reduced data to describe the investigated processes in the target/sample.
  \item[Example:]  Expand-Maximize-Compress (EMC)  for orientation, Difference-Map for phasing \cite{Loh2009, s2e_recon_bitbucket}.
\end{description}

\section{D4.1: Short pulse laser-matter interaction\label{sec:short_pulse}}
%
\subsection{Introduction}
Short pulse lasers typically deliver laser pulses in the infrared (800~nm to 1000~nm) at pulse durations below one picosecond. Todays high power lasers can reach intensities of up to $10^{21}~\text{W}/\text{cm}^2$ on spot sizes of a few microns and pulse durations on the order of a few ten femtoseconds.

Despite their intensity, these sources usually do not penetrate a solid density target but rather create a plasma at the target front side, accelerating electrons to relativistic energies in the strong electric field of the laser and pushing them into the target via the $\vec{v}\times\vec{B}$ force once the velocity $v$ approaches the speed of light.

The generation of relativistic electrons at the front side, the transport of electrons through the target and the subsequent formation of a sheeth of electrons at the target rear side all happen on time scales below a few hundred femtoseconds. They can create plasma instabilities, ionize and heat the target bulk, generate strong magnetic fields or drive shocks inside the target.

These phenomena can potentially be studied with high spatial resolution of a few nanometer and temporal resolution of a few femtosecond using X-Ray lasers such as the European XFEL.

The particle-in-cell method is the most advanced simulation method to study the interaction of a high-power laser with a solid density target. Realistic 3D particle-in-cell simulations of laser-irradiated solid density plasmas require Petascale computing capabilities and produce hundreds of Terabytes of data.

Within \texttt{simex\_platform} we interface particle-in-cell (PIC) codes such as \texttt{PIConGPU} to XFEL sources to allow for synthetic scattering simulations on the free and, in the future, bound electrons in the solid density plasma simulated by the PIC codes. Interfaces are based on the openPMD meta data format.

As a first test of the capabilities of the \texttt{simex\_platform} free electron density data from a PIConGPU simulation of the interaction of a short-pulse laser system\footnote{see details below} available at the High Energy Density (HED) instrument \cite{Nakatsutsumi2014} at the European XFEL will be used to compute a synthetic scattering image.
%
\subsection{X-ray source}
The x-ray source in this virtual experiment is the SASE2 beamline at the European XFEL, Hamburg, Germany \cite{Tschentscher2011}.
X-ray pulses are simulated using the code FAST \cite{Saldin1999} developed at DESY. Starting from parameters of the electron linear accelerator
(e.g. electron energy and bunch charge), FAST simulates the self amplification of stimulated emission (SASE) process \cite{} in the FEL's
undulator. As a result, the simulation returns the wavefront (i.e. the complex electric field components
along the horizontal and vertical polarization axis) in the plane of the undulator exit. In the coordinate frame of the
XFEL, this plane has the longitudinal coordinate $z=0$. For practical reasons and because FAST is not a public domain software,
the FAST wavefront data are read from the x-ray pulse database \cite{xpd_xfel}.

The following x-ray pulse parameters are available:
\begin{description}
  \item[Photon energy:] 6 - 12 keV. The targeted initial photon energy at the European XFEL is 6 keV.
  \item[energy spread:] SASE operation: 10$^{-3}$, seeded operation: $10^{-5\dots-4}$.
  \item[Pulse duration:] of the order 10 fs.
  \item[Number of photons] of the order 10$^{12}$.
\end{description}
%
\subsection{Optical laser source}
The optical laser source is described through a set of parameters characterizing the short-pulse pump-probe (``PP'') laser at the HED
instrument:
\begin{description}
  \item[Wavelength:] 800 nm
  \item[Pulse duration:] bandwidth limited to 15 - 80 fs.
  \item[Intensity:] $1\times 10^{17}\,\text{W}/\,\text{cm}^2$
  \item[Pulse energy:] 2 mJ
  \item[Focus:] $3\,\mu\text{m}$
  \item[Profile:] Gaussian in t and x,y
\end{description}
%
\subsection{Propagation\label{sec:short_pulse_prop}}
Propagation of XFEL pulses is modeled through the existing XFELPhotonPropagator calculator. The latter is an interface
to \texttt{WPG} \cite{Samoylova2016_submitted, wpg_github}, a python high level interface for the \texttt{Synchrotron Radiation Workshop} (SRW) \cite{Chubar2008, srw_github}.
The simulation parameters reflect the conditions at the High Energy Density (HED) instrument \cite{Nakatsutsumi2014}
at the European XFEL. The focal spotsize can be fixed at three different ranges,
$\approx 200 \mu\text{m}$, $\approx 20 \mu\text{m}$, and $\approx 2 \mu\text{m}$ \marginpar{Fix numbers, how large are the ranges?}
%
\subsection{Photon-Matter interaction}
The short-pulse laser-plasma interaction is modeled with \texttt{PIConGPU} \cite{Bussmann2013, picongpu_github}.
\texttt{simex\_platform} provides tools for the conversion of wavefronts into photon distributions.
\subsection{Scattering}
\ldots \marginpar{Fill in, HZDR?}
\subsection{Detection}
At an early stage, the detector will be modeled as an ideal x-ray detector, i.e. registering the exact number of incident photons in each
pixel as simulated by the scattering module. At a later stage, the realistic detector response will be modeled within the
X-ray Camera Simulation Toolkit \cite{Joy2015}.% as described in \cite{Rueter2016_submitted}.
%
\subsection{Data analysis}
From the small-angle x-ray scattering data one can infer important information about the structure of the highly excited target during
the interaction with the optical laser. E.g. the electron density is directly encoded in the diffraction pattern. In order to
reconstruct the electron density from the diffraction data, the complex phase of the probe has to be reconstructed using an iterative phasing
algorithm. Such algorithms are widely available, e.g. \ldots \marginpar{Fill in}.
The so calculated electron density can then be compared to the particle data in the PIC simulation to assess the reliability of
the phasing algorithm under given realistic experimental condition. \marginpar{What else would be of interest here?}

\section{D4.2: Long pulse laser-matter interaction\label{sec:long_pulse}}
\subsection{Introduction}
TBD
\subsection{X-ray source}
\subsection{Optical laser source}
The optical laser source is described through a set of parameters:
\begin{description}
  \item[Wavelength:]
  \item[Pulse duration:]
  \item[Intensity:]
  \item[Pulse energy:]
  \item[Focus:]
  \item[Profile:]
\end{description}
\subsection{Propagation}
Propagation of XFEL pulses is modeled through the existing XFELPhotonPropagator calculator as described in Sec.~\ref{sec:short_pulse_prop}.
\subsection{Photon-Matter interaction}
The photon matter interaction is modeled via radiation hydrodynamics simulation using the code Multi. Data format converters are part of
\texttt{simex\_platform}.
\subsection{Scattering}
TBD
\subsection{Detector}
TBD
\subsection{Data analysis}

\printbibliography
\end{document}



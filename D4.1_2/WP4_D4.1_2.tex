\documentclass[a4paper]{article}
\usepackage[]{graphicx}
\usepackage{latexsym}
\usepackage[centertags]{amsmath}
\usepackage{amssymb}

%%%%%%%%%%%%%%%%%%%%%%%%%%%%%%%%%%%%%%%%%%%%%%%
%   BIBLIOGRAPHY SETTINGS                     %
%%%%%%%%%%%%%%%%%%%%%%%%%%%%%%%%%%%%%%%%%%%%%%%
\usepackage[bibstyle=nature,sorting=none,=maxnames=1000,eprint=false, defernumbers=true, backend=biber]{biblatex}
\usepackage{hyperref}

\renewcommand*\finalnamedelim{, and\addspace}
\DeclareNameAlias{sortname}{last-first}
\renewcommand{\newunitpunct}{, }

\AtEveryBibitem{%
  \clearfield{day}%
  \clearfield{month}%
  \clearfield{endday}%
  \clearfield{endmonth}%
  \clearfield{issn}%
  \clearfield{issue}%
}
%convert titles to hyperlinks using doi
\ExecuteBibliographyOptions{doi=false}
\newbibmacro{string+doi}[1]{%
  \iffieldundef{doi}{#1}{\href{http://dx.doi.org/\thefield{doi}}{#1}}}
\DeclareFieldFormat*{title}{\usebibmacro{string+doi}{\mkbibemph{#1}}}

%\addbibresource{/home/grotec/Documents/LiteratureDB/bibtex/jabref.bib}
\addbibresource{references.bib}
%%%%%%%%%%%%%%%%%%%%%%%%%%%%%%%%%%%%%%%%%%%%%%%
% END BIBLIOGRAPHY SETTINGS                   %
%%%%%%%%%%%%%%%%%%%%%%%%%%%%%%%%%%%%%%%%%%%%%%%


\title{Design Report and Advanced Simulation Software for Laser -- Matter interaction}
%In fullfilment of deliverables D4.1 and D4.2 in EUCALL SIMEX
\author{Carsten Fortmann-Grote, Axel Huebl, Michael Bussmann } % Add yourself.
\date{\today}

\begin{document}
\maketitle
\section{Introduction}
This document describes the design and implementation of simulation software for interaction of intense, coherent radiation from x-ray free
electron lasers, synchrotrons, and optical lasers with solid, liquid, or gasous samples and targets.
It discusses the generic simulation platform within which the simulations will be performed and the
concrete simulation tools for two virtual experiments:
\begin{enumerate}
  \item Interaction of high power, \textbf{short} pulse optical laser radiation (duration $\approx 10\,\text{fs}$)
    with matter and subsequent probing of the resulting state by intense x-ray pulses,
  \item Interaction of high energy, \textbf{long} pulses of optical laser light (duration $\approx 10\,\text{ps}$
    with a solid target and probing with x-rays.
\end{enumerate}%

\section{The SIMEX simulation platform \label{sec:simex_platform}}
%
\subsection{Introduction}
The computer program \texttt{simex\_platform}
\cite{simex_github}
is a platform for the simulation of experiments at advanced laser light sources. It allows the
user to assemble a virtual experiment through combination of suitable programs for the light source (e.g. a synchrotron, an x-ray free electron
laser or an optical laser), beam transport from the source to the sample or target, interaction of the light with the sample or target,
propagation of the scattered light behind the sample or target, and detection in a light detector. \texttt{simex\_platform} comes preloaded with
a number of such \textit{Calculators}, aimed at the simulation of various typical laser light experiments. Furthermore, researchers
can replace individual built-in \textit{Calculators} by their own codes. In this way, they can embed their codes
into a more realistic simulation environment compared to running the code in stand-alone fashion with more or less idealized parameters.
%
\subsection{Building blocks of a virtual photon experiment}
With slight variations, photon based experiments can be broken down into six individual blocks: The photon source, the photon transport
from the source to the experiment, the interaction of photons and matter, propagation of scattered and transmitted radiation from the
target/sample to the detector and photon detection in a detector, followed by data analysis.
In \texttt{simex\_platform} each of these blocks is
represented by an \textit{Abstract Calculator}, which defines virtual methods for communication between subsequent calculators and execution of
the underlying implementation of a particular algorithm.
These virtual methods can be used e.g. in a workflow manager to execute the virtual experiment.
Specialized \textit{Calculators}, derived from the\textit{Abstract Calculator}, implement one particular stage of the simulation, using a
specific method or algorithm. Data
format conversion, execution statements, and submission of calculation tasks to the operating system, are handled on this
level. The
\textit{Calculators} are the algorithmic building blocks describing subsequent stages in the beamline.
%
\subsubsection{Photon source}
The mechanism to generate the radiation before any optical elements and before any photon-matter interaction has happened.
\begin{description}
  \item[Abstract calculator:] AbstractPhotonSource
  \item[Physics:] Depending on nature of the source: radiation from charged particle acceleration, spontanuous emission, laser medium.
  \item[Input data:] Parameters of the photon source, e.g. for an undulator: electron bunch charge, electron energy, undulator period and length.
  \item[Output data:] Representation of the light source (e.g. as a wavefront, rays, photon distribution)
  \item[Example method:] The code FAST \cite{Saldin1999} generates 3D (x-y-t) wavefronts at the exit of the undulator in an x-ray free electron laser.
\end{description}
%
\subsubsection{Photon Propagation}
Propagates the radiation as described by the photon source calculator from the source to the point of interaction with the sample or target under
    investigation. Describes focussing, filtering, pulse shaping,
    and other optical effects realized through lenses, mirrors, apertures, grids etc.
\begin{description}
  \item[Abstract calculator:] AbstractPhotonPropagator
  \item[Physics:] Propagation of light through optical elements.
  \item[Input data:] Wavefront or rays at beginning of beamline.
  \item[Output data:] Wavefront or rays at target/sample interaction point.
  \item[Example method:] Fourier optics wavefront propagation, ray tracing.
\end{description}
%
\subsubsection{Photon Interactor}
Interaction of the photons with the target or sample. Takes into account elementary processes like absorption, emission, scattering of radiation and secondary processes like collisional ionization and recombination. The end product is the electronic state of the sample/target as a function of time during the interaction with the external light source.
\begin{description}
  \item[Abstract calculator:] AbstractPhotonInteractor
  \item[Physics:] Absorption, emission, and scattering of radiation by charged particles. Secondary processes like electron impact ionization, three
    body recombination. Acceleration of particles in external fields and backreaction of excited matter on the radiation.
  \item[Input data:] Wavefront or rays at sample/target interaction point.
  \item[Output data:] Snapshots of electronic state trajectory during the radiation exposure. E.g. electron density, form factors, electronic
    wavefunction or density matrix.
  \item[Example method:] Molecular Dynamics, Particle-in-cell, Ab-initio.
\end{description}
%
\subsubsection{Photon Diffractor}
Typically in a photon experiment, the scattered or transmitted radiation, or reaction products from the photon-matter
interaction, such as photo-electrons, are detected and used to infer information about the sample/target.
This module is named \textit{Diffractor} although diffraction is by far not the only anticipated diagnostic method to be simulated.
\begin{description}
  \item[Abstract calculator:] AbstractPhotonDiffractor
  \item[Physics:] Absorption, emission, and scattering of radiation by the target/sample.
  \item[Input data:] Wavefront or rays at sample/target interaction point and electronic state trajectory.
  \item[Output data:] Radiation signal at a given position of a radiation detector.
  \item[Example methods:] Born approximation scattering from single molecules, small-angle x-ray scattering (SAXS), x-ray Thomson
    scattering (XRTS).
\end{description}
%
\subsubsection{Photon Detector}
The scattered radiation will ultimately  be collected in a detection device. The Photon Detector models the response of the detector to the incoming
radiation.
\begin{description}
  \item[Abstract calculator:] AbstractPhotonDetector
  \item[Physics:] Response of the detector to incoming radiation, e.g. photoelectron conversion, charge transport, and counting in a photon detector.
  \item[Input data:] Intensity or photon distribution at the detector.
  \item[Output data:] Detector response to the signal, e.g. photons per pixel in an area detector.
  \item[Example method:] X-ray camera simulation toolkit X-CSIT \cite{Joy2015}.
\end{description}
%
\subsubsection{Photon Analysis}
Following the detector readout, the recorded raw data has to be analyzed in order to extract the desired information about the target/sample. The
specific algorithms depend largely on the details of the experiment and the underlying questions. Here, we discuss the analysis steps to be taken
in a single particle diffractive imaging experiment.
\begin{description}
  \item[Abstract calculator:] AbstractPhotonAnalyzer
  \item[Physics:]  Analysis of raw detector data. In SPI, the electron density of the sample is reconstructed by assembling a 3D diffraction volume
    from measured 2D diffraction patterns and subsequent phasing of the data to solve the inverse scattering problem.
  \item[Input data:] Detector response (raw data)
  \item[Output data:] Reduced data to describe the investigated processes in the target/sample.
  \item[Example:]  Expand-Maximize-Compress (EMC)  for orientation, Difference-Map for phasing \cite{Loh2009, s2e_recon_bitbucket}.
\end{description}

\section{D4.1: Short pulse laser-matter interaction\label{sec:short_pulse}}
%
\subsection{Introduction}
High power short pulse lasers typically deliver laser pulses in the infrared (800~nm to 1000~nm) at pulse durations below one picosecond. Todays high power lasers~\cite{Siebold2008} can reach intensities of up to $10^{21}~\text{W}/\text{cm}^2$ on spot sizes of a few microns and pulse durations on the order of a few ten femtoseconds.

Despite their intensity, these sources usually do not penetrate a solid density target but rather create a plasma at the target front side, accelerating electrons to relativistic energies in the strong electric field of the laser~\cite{Kluge2011} and pushing them into the target via the $\vec{v}\times\vec{B}$ force once the velocity $v$ approaches the speed of light~\cite{Mulser2010,Gibbon1996}.

The generation of relativistic electrons at the front side, the transport of electrons through the target and the subsequent formation of a sheeth of electrons at the target rear side all happen on time scales below a few hundred femtoseconds~\cite{Macchi2013}. They can create plasma instabilities~\cite{Metzkes2014}, ionize and heat the target bulk~\cite{Huang2013}, generate strong magnetic fields or drive shocks inside the target.

These phenomena can potentially be studied with high spatial resolution of a few nanometer and temporal resolution of a few femtosecond using X-Ray lasers~\cite{Kluge2014,Kluge2016} such as the European XFEL.

The particle-in-cell method~\cite{Birdsall2004} is the most advanced simulation method to study the interaction of a high-power laser with a solid density target. Realistic 3D particle-in-cell simulations of laser-irradiated solid density plasmas require Petascale computing capabilities and produce hundreds of Terabytes of data~\cite{ornl_picongpu}.

Within \texttt{simex\_platform} we interface particle-in-cell (PIC) codes such as \texttt{PIConGPU}~\cite{Bussmann2013, picongpu_github} to XFEL sources to allow for synthetic scattering simulations on the free and, in the future, bound electrons in the solid density plasma simulated by the PIC codes. Interfaces are based on the openPMD~\cite{openPMD} meta data format.

As a first test of the capabilities of the \texttt{simex\_platform} free electron density data from a PIConGPU simulation of the interaction of a short-pulse laser system\footnote{see details below} available at the High Energy Density (HED) instrument \cite{Nakatsutsumi2014} at the European XFEL will be used to compute a synthetic scattering image.
%
\subsection{X-ray source}
The x-ray source in this virtual experiment is the SASE2 beamline at the European XFEL, Hamburg, Germany \cite{Tschentscher2011}.
X-ray pulses are simulated using the code FAST \cite{Saldin1999} developed at DESY. Starting from parameters of the electron linear accelerator
(e.g. electron energy and bunch charge), FAST simulates the self amplification of stimulated emission (SASE) process \cite{} in the FEL's
undulator. As a result, the simulation returns the wavefront (i.e. the complex electric field components
along the horizontal and vertical polarization axis) in the plane of the undulator exit. In the coordinate frame of the
XFEL, this plane has the longitudinal coordinate $z=0$. For practical reasons and because FAST is not a public domain software,
the FAST wavefront data are read from the x-ray pulse database \cite{xpd_xfel}.

The following x-ray pulse parameters are available:
\begin{description}
  \item[Photon energy:] 6 - 12 keV. The targeted initial photon energy at the European XFEL is 6 keV.
  \item[energy spread:] SASE operation: 10$^{-3}$, seeded operation: $10^{-5\dots-4}$.
  \item[Pulse duration:] of the order 10 fs.
  \item[Number of photons] of the order 10$^{12}$.
\end{description}
%
\subsection{Optical laser source}
The optical laser source is described through a set of parameters characterizing the short-pulse pump-probe (``PP'') laser at the HED
instrument:
\begin{description}
  \item[Wavelength:] 800 nm
  \item[Pulse duration:] bandwidth limited to 15 - 80 fs.
  \item[Intensity:] $1\times 10^{17}\,\text{W}/\,\text{cm}^2$
  \item[Pulse energy:] 2 mJ
  \item[Focus:] $3\,\mu\text{m}$
  \item[Profile:] Gaussian in t and x,y
\end{description}
%
\subsection{Propagation\label{sec:short_pulse_prop}}
Propagation of XFEL pulses is modeled through the existing XFELPhotonPropagator calculator. The latter is an interface
to \texttt{WPG} \cite{Samoylova2016_submitted, wpg_github}, a python high level interface for the \texttt{Synchrotron Radiation Workshop} (SRW) \cite{Chubar2008, srw_github}.
The simulation parameters reflect the conditions at the High Energy Density (HED) instrument \cite{Nakatsutsumi2014}
at the European XFEL. The focal spotsize can be fixed at three different ranges,
$\approx 200 \mu\text{m}$, $\approx 20 \mu\text{m}$, and $\approx 2 \mu\text{m}$ \marginpar{Fix numbers, how large are the ranges?}
%
\subsection{Photon-Matter interaction}
The short-pulse laser-plasma interaction is modeled with \texttt{PIConGPU} \cite{Bussmann2013}.

\texttt{PIConGPU} is an Open Source~\cite{picongpu_github} explicit, relativistic 3D3V particle-in-cell (\texttt{PIC}) code which can simulate the interaction of high power, ultra-short laser pulses with matter.

High power laser pulses, \texttt{HPL}, are defined by their time dependent magnetic and electric field components and solved on a regular mesh (usually Cartesian) via a so called Maxwell solver using e.g. finite difference time domain techniques~\cite{Yee1966}.

Different to the \texttt{HPL}, the X-ray interaction is modeled via a Monte-Carlo photon interaction model. Each photon is described by its wave vector and a phase\footnote{In the future, the photon model will also include polarization.}.

An X-ray pulse described by temporally and spatially varying intensities and wavefronts will have to be converted into a time-depended spatial distribution of photons. These photons are then tracked through the volume simulated by \texttt{PIConGPU}.

Photons are scattered according to predefined scattering functions that depend on the local properties of the matter irradiated by the HPL. In the most simple case, the local free electron density is used as an input for calculating the probability for Thomson scattering of the photons off free electrons.

\texttt{simex\_platform} provides tools for the conversion of wavefronts into photon distributions.
\subsection{Scattering}

There are several methods that already are or will be implemented for scattering. We list them below in order of ascending generality (and technical difficulty).

\begin{enumerate}
\item 3D Fast Fourier Transform of electron density \label{pmi:methods:fft}
\item Ex-situ scattering using stand-alone Monte-Carlo photon scattering \label{pmi:methods:exsituphoton}
\item In-situ scattering using a modified particle-in-cell algorithm and far field Lienard-Wiechert potential calculation.\label{pmi:methods:insitulienardwiechert}
\item In-situ scattering using Monte-Carlo photon scattering\label{pmi:methods:insituphoton}
\item In-situ scattering using Monte-Carlo photon scattering with absorption and emission of photons\label{pmi:methods:insituphotoninteract}
\end{enumerate}

Ex-situ Fast Fourier Transforms have been implemented using the \texttt{LiFFT} library \cite{liblifft_github} and can be performed in a post processing step on \texttt{openPMD} outputs of the electron density. Implementation in \texttt{simex\_platform} is currently under discussion.

Ex-situ scattering is performed by the \texttt{XRT}~\cite{xrt_github} code. The code will soon be released as Open Source. It uses the same libraries and techniques as \texttt{PIConGPU}. Specifically, it can read \texttt{openPMD}~\cite{openPMD} files via \texttt{libSplash}~\cite{libSplash_github} and traces photons through a mesh using \texttt{libPMacc}~\cite{picongpu_github}. One can define an arbitrary interaction using a C++ function object. The current code version provides Thomson Scattering from free electrons based on the local free electron density. Photon scattering is computed via repeated Monte-Carlo evaluation of the scattering function, allowing for multiple scattering.

In-situ scattering using Lienard-Wiechert potentials will be based on an already existing in-situ Lienard-Wiechert potential calculator plugin to \texttt{PIConGPU} \cite{Pausch2013}. This method will serve mainly as a cross check to the other methods. It involves a modification of the PIC algorithm that will be implemented in the future. The method allows to compute arbitrary angular far field scattering spectra during the interaction of the HPL with matter.

An implementation of in-situ Monte-Carlo photon scattering in \texttt{PIConGPU} would be based on the techniques developed for the \texttt{XRT} code. Here, the main issue is the large memory consumption by the photons traced while the simulation is running.

If in-situ photon scattering can be succesfully implemented, interaction of the X-ray photons with the ions and electrons in the simulation can be implemented. Possible scenarios include elastic scattering of photons from electrons, excitation of ions, absorption and emission of photons. Finally, other processes that produce X-ray photons could be included, such as deexcitation, Bremsstrahlung, etc. These could help to determine backgrounds. In summary, such a framework of physical models would allow for fully kinetic in-situ radiation transport modeling in a particle-in-cell code.

As discussed, development of these capabilities are ongoing and will be subsequently made available in \texttt{simex\_platform} and \texttt{PIConGPU}. Compatibility of all software packages will be ensured by reuse of existing libraries and the common meta data format \texttt{openPMD}.

\subsection{Detection}
At an early stage, the detector will be modeled as an ideal x-ray detector, i.e. registering the exact number of incident photons in each
pixel as simulated by the scattering module. At a later stage, the realistic detector response will be modeled within the
X-ray Camera Simulation Toolkit \cite{Joy2015}.% as described in \cite{Rueter2016_submitted}.
%
\subsection{Data analysis}
From the small-angle x-ray scattering data one can infer important information about the structure of the highly excited target during
the interaction with the optical laser. E.g. the electron density is directly encoded in the diffraction pattern. In order to
reconstruct the electron density from the diffraction data, the complex phase of the probe has to be reconstructed using an iterative phasing
algorithm. Such algorithms are widely available, e.g. \ldots \marginpar{Fill in}.
The so calculated electron density can then be compared to the particle data in the PIC simulation to assess the reliability of
the phasing algorithm under given realistic experimental condition. \marginpar{What else would be of interest here?}

\section{D4.2: Long pulse laser-matter interaction\label{sec:long_pulse}} %%%%%%%%%%%%%%%%%%%%%%ESRF
\subsection{Introduction}
Long pulse laser systems are capable of reaching intensities of up to $\sim 10^{12}$ -- 10$^{15} \text{ W}/\text{cm}^2$ with pulse lengths of greater than a few nanoseconds. The pulse length and focal spot of the laser can depend heavily on the sample material and target package. Generally, focal spot sizes range between $\sim$ 100 $\mu$m up to 1 mm. With these specifications, the laser energies now required to reach the highest intensities are on the order of hundreds of joules.

Temporally shaped laser pulses interacting with overdense target material create a rapidly expanding plasma that
can generate a shockwave in the ablating material. This shock wave can travel through the target at several km/s
and compress the sample material to pressures reaching several hundred GPa. (For comparison, the Earth's core
pressure is $\sim$ 330 GPa). By using a ramp temporal pulse, where the laser intensity is slowly increased, the temperatures generated remain much cooler than during rapid shock compression and the solid state of matter can be studied up to several TPa (1 TPa = 1000 GPa).

High-power \marginpar{not "high energy" ?} laser facilities are now becoming increasingly in demand at new
generation x-ray light sources (X-FELs, 3rd generation synchrotron sources). The potential of long pulse laser experiments are already yielding extreme conditions of matter well beyond the reach of static high-pressure techniques and combinations with X-ray techniques provide enticing experiments at new extreme states of matter.

We design here simulations for a prototypical experiment combining a high-energy laser system and x-ray radiation
from a 3rd generation synchrotron. The laser shock-compresses a tailored solid density target and the compressed
matter is subsequently probed by x-rays. X-ray absorption spectroscopy (XAS) allows to monitor the compression and
to characterize the electronic and structural states in the target. Through variation of the delay between optical
laser pulse and x-ray probe pulse, time resolved data is obtained.

\subsection{X-ray source}
\subsubsection{European Synchrotron Radiaton Facility (ESRF)}
For X-ray absorption studies at ESRF, an intense source of linearly polarized X-rays, with photon energy tunable
between $\sim$ 5 and 28 keV, is used at the ID24 beamline. Beamline optics focus the X-ray beam to a 3 $\times$ 3
$\mu$m (H $\times$ V) FWHM. However, the horizontal spot size is energy dependent and ranges from 3 $\mu$m at 5 keV
to 50 $\mu$m at 28 keV. The X-ray flux on sample ranges from 1 $\times$ 10$^{14}$ photons s$^{-1}$ at 7 keV to 4
$\times$ 10$^{13}$ photons s$^{-1}$ at 28 keV. The X-ray source at the ESRF synchrotron runs in several timing
modes, several of which are suitable for laser shock experiments where a single x-ray bunch ($\sim$ 100 ps) is used
to probe the sample during a highly compressed state. An overview of the time-resolved and extreme-conditions XAS
(TEXAS) facility is presented in \cite{Pascarelli2016}.
\subsection{Optical laser source}
\subsubsection{ESRF}
The optical laser source, used at the ESRF for shock compression experiments in May/June 2016,  is described through a set of parameters:
\begin{description}
  \item[Wavelength:] 1053 nm
  \item[Pulse duration:] 4 ns - 15 ns
  \item[Intensity:] $2\times 10^{13}\,\text{W}/\,\text{cm}^2$
  \item[Pulse energy:] $\textless \, 30\, \text{J}$
  \item[Focus:] 100 or 300 $\mu\text{m}^{2}$
  \item[Profile:] Gaussian in x,y and `flat top' in $t$
\end{description}
A call for tender will be issued in 2016 for a new high-energy laser system to be installed at the ESRF, with increased energy ($\textless$ 200 J) and frequency doubling crystals to take advantage of the 2$\omega$ at $\sim$ 527 nm.
\subsection{Propagation}
Propagation of XFEL pulses is modeled through the existing XFELPhotonPropagator calculator as described in Sec.~\ref{sec:short_pulse_prop}.
\subsection{Photon-Matter interaction}
Nearly all of the physical processes of long pulse laser-matter interactions are described by partial differential
equations that can be solved with careful construction of numerical simulation codes. Here, the laser-matter
interactions are modelled with radiation-hydrodynamics computer codes ``Esther'' \cite{Esther} or ``MULTI2D''
\cite{Ramis2009}.

For the 1D hydrocodes, a moving mesh is applied in an arbitrary Lagrangian-Eulerian framework where the coordinates of
the mesh contain the variables from which density can be calculated (mass of each zone is fixed). The most important variables
considered in these codes are those associated with the extreme conditions generated by the high-power lasers: pressure (density),
temperature and velocity. Feedback of these hydrocode outputs are crucial in the design and implementation of laser shock/ramp
experiments with X-ray interactions.

The X-ray probe duration, during laser shock compression, can range from nanosecond down to femtosecond exposures.
It is crucial to have an understanding of shock transit times so that accurate timing of the X-ray probe,
with respect to the laser initialisation, can be made. The final density state reached by a shock, as calculated
from the hydrocode simulation packages, can define the shock velocity.

\subsubsection{Esther}
The Esther hydrocode was written by Patrick Combis and Laurent Videau of the CEA, Paris, France.
\footnote{The Esther hydrocode is currently presented in French only.}
A license to use Esther can be obtained by requesting access (via email) from the authors (for academic use only).
As part of the \texttt{simex\_platform}, a user interface is being developed to create the input files necessary for running
the hydrocode simulations and to write the output data to a openPMD conform hdf5 file.

\subsubsection{MULTI}
In the MULTI2D hydrocode \cite{Ramis2009},
the hydrodynamic equations solved by the code are combined with a multigroup method for radiation transport.

\subsection{Scattering}
TBD
\subsection{Detector}
TBD
\subsection{Data analysis}

\printbibliography
\end{document}


